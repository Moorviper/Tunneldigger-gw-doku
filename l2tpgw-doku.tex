\documentclass[11pt, a4paper,ngerman]{article}
\usepackage{basicff}

\usetikzlibrary{patterns} % preamble
\tcbuselibrary{skins} % preamble

\usepackage{tikz}
\usepackage{PTSansNarrow}
\usetikzlibrary{matrix}

\usepackage{pgfplots}
\pgfplotsset{
 compat=newest
  }
\tcbset{colframe=red!75!black}
\newenvironment{proggen}{\begin{center}}{\end{center}}

\usepackage{array}


\newcolumntype{C}[1]{>{\centering\arraybackslash}m{#1}}

\oddsidemargin-10mm
\title{
\color{white}
 $\bullet$ \\ $\bullet$ \\ $\bullet$ \\
 \color{black}
 % \color{white}
 % $\bullet$ \\
 \color{black}
L2TP Gateway Doku\\
Tunneldigger
 \begin{center}
	% \includegraphics[scale=0.5]{./pictures/wohnzimmer.png}
\end{center}
 % \includegraphics{./pictures/wohnzimmer.png}
}
% \includegraphics{./pictures/asrock.png} Q1900M \\ \color{white} $\bullet$ \\ $\bullet$ \\ $\bullet$ \\ \color{black} \\ \apple \\ 10.10.3

\author{Daniel Krah}
% \date{1.6.2015}

\begin{document}
% \AddToShipoutPicture{\BackgroundPic}
\maketitle%
\newpage%
 \tableofcontents%
% \newpage
%==================================================================================
% \begin{center}
% \textbf{Vorwort}
% \end{center}



%
% \input{hardware.tex}







%  To Do
% \input{todo.tex}
\newpage
\section{Überblick Installation eines Tunneldigger-Gateways bei Online.net}
\subsection{Benötigt:}
\subsubsection{Hardware}
  \begin{enumerate}
    \item Server mit schnellem garantiertem Upload\\
          In diesem Fall eine Dedibox SC mit 2,5 Gbit \\
          (ca 380 Mbit Upload dauerhaft verfügbar)
    \item Einen Uplink ans Backbone des Freifunk Rheinland da der Prozessor zu schwach ist um mehr als 35 Mbit über openVPN zu drücken.  
  \end{enumerate}

\subsubsection{Software}
Softwareseitig werden folgende Pakete/Kernelmodule verwendet:
\begin{enumerate} 
   \item  Ubuntu 16.04 LTS 
   \item  Batman-adv (Kernelmodul -> Einfach laden)
   \item  isc-dhcpd (DCHP-Server für IPv4 Adressen)
   % \item  radav (DHCP-Server für IPv6)
\end{enumerate}

% \subsection{Vorraussetzung}

% \begin{itemize} 
%    \item Netzwerkinterface ins Internet eth0
%    \item Netzwerkinterface für VPN mesh-vpn
%    \item Netzwerkinterface für Batman: bat0
%    \item Client IPv4 Netz 10.185.8.0/21 
% \end{itemize}

\section{Tunneldigger}
\subsection{Was ist das Ziel ?}

\monocodebox{sh}{ifconfig}{./command/ifconfigVpnBatTD.sh}{false}{1}{9999999}
\monocodebox{sh}{brctl}{./command/brctlEmpty}{false}{1}{9999999}

\subsection{Welche Kernelmodule müssen geladen werden}
\monocodebox{sh}{/etc/modules}{./etc/modules}{false}{1}{9999999}

\subsection{Die Tunneldigger Bridge}
\monocodebox{sh}{/etc/network/interfaces.d/tunneldigger}{./etc/network/interfaces.d/tunneldigger}{false}{1}{9999999}

\subsubsection{Starten des Brokers}
\monocodebox{sh}{/srv/tunneldigger/start-broker.sh}{./srv/tunneldigger/start-broker.sh}{false}{1}{9999999}
 
\subsubsection{Beim Aufbau einer Verbindung}
\monocodebox{sh}{/srv/tunneldigger/scripts/session-up.sh}{./srv/tunneldigger/scripts/session-up.sh}{false}{1}{9999999}

\subsubsection{Beim Abbau einer Verbindung}
\monocodebox{sh}{/srv/tunneldigger/scripts/session-pre-down.sh}{./srv/tunneldigger/scripts/session-pre-down.sh}{false}{1}{9999999}

\subsubsection{Beim Boot}
\monocodebox{sh}{/usr/local/bin/bat-startup.sh}{./usr/local/bin/bat-startup.sh}{false}{1}{9999999}

\subsubsection{Beim Boot}
\monocodebox{sh}{/usr/local/bin/bat-startup.sh}{./usr/local/bin/bat-startup.sh}{false}{1}{9999999}

\subsubsection{DHCP-Server für IPv4:}
\monocodebox{sh}{/etc/dhcp/dhcpd.conf}{./etc/dhcp/dhcpd.conf}{false}{1}{9999999}


% \subsection{Konfiguration von AirVPN}

% sudo apt-get install openvpn

% Bei Air VPN eine Konfig datei erstellen und  in /etc/openvpn/airvpn.conf speichern.
% Als standard biegt die config aber alle routen auf das VPN Interface um. Damit ist der Server hinter einem NAT und per SSH nichtmehr zu erreichen. 
% Deshalb sollten die Routen automatisch erstellt werden und die Zeile
% route-delay 5 <- bei mir war es in der airvpn-config nicht drin...
% auskommentiert und

% route-noexec

% ergänzt werden.

% ein 
% sudo service openvpn start 
% startet den Tunnel
% sudo update-rc.d openvpn defaults
% startet den Tunnel bei jeden boot.





% \monocodebox{sh}{/etc/network/interfaces.d/tunneldigger}{./etc/network/interfaces.d/tunneldigger}{false}{1}{9999999}


% \monocodebox{sh}{/srv/tunneldigger/scripts/session-up.sh}{./srv/tunneldigger/scripts/session-up.sh}{false}{1}{9999999}

% \monocodebox{sh}{ifconfig}{./command/ifconfigVpnBatTD.sh}{false}{1}{9999999}











% \input{kap1.tex}









%=========================================================================================================================
%
%-------------END
\end{document}
