% arara: lualatex: { shell: yes, action: nonstopmode, synctex: yes}
% arara: lualatex: { shell: yes, action: nonstopmode, synctex: yes}
% asarara: lualatex: { shell: yes, action: nonstopmode, synctex: yes,  options: "-output-directory=_build"}
% asarara: lualatex: { shell: yes, action: nonstopmode, synctex: yes,  options: "-output-directory=_build"}
\documentclass{article}
\usepackage[ngerman]{babel}
\usepackage[no-math]{fontspec}

\usepackage{mwe}
\usepackage{luacode}
\usepackage{shellesc}
% \documentclassw[11pt, a4paper,ngerman]{article}
\usepackage{basicff}

\usetikzlibrary{patterns} % preamble
\tcbuselibrary{skins} % preamble

\usepackage{tikz}
% \usepackage{PTSansNarrow}
\usetikzlibrary{matrix}

\usepackage{pgfplots}
\pgfplotsset{
 compat=newest
  }
\tcbset{colframe=red!75!black}
\newenvironment{proggen}{\begin{center}}{\end{center}}

\usepackage{array}
% \setmainfont[Path=/Applications/Microsoft Word.app/Contents/Resources/Fonts/]{Calibri.ttf}
% \setsansfont[Path=/Applications/Microsoft Word.app/Contents/Resources/Fonts/]{Calibri.ttf}
% \setmonofont[Path=/Applications/Microsoft Word.app/Contents/Resources/Fonts/]{Calibri.ttf}
% \usepackage{datetime}
% \pagestyle{fancy}


\setmainfont{UbuntuL.ttf}
\setsansfont{UbuntuR.ttf}
\setmonofont{UbuntuMonoR.ttf}

\newcolumntype{C}[1]{>{\centering\arraybackslash}m{#1}}

\oddsidemargin-10mm
\title{
\color{white}
 $\bullet$ \\ $\bullet$ \\ $\bullet$ \\
 \color{black}
 % \color{white}
 % $\bullet$ \\
 \color{black}
 Mitschrift \\
Datenbanken \\
Peinl \\
\color{white}
$\bullet$ \\
\color{black}
 \begin{center}
	% \includegraphics[scale=0.5]{./pictures/CaptainWarschburger.jpg}
\end{center}
 % \includegraphics{./pictures/wohnzimmer.png}
}
% \includegraphics{./pictures/asrock.png} Q1900M \\ \color{white} $\bullet$ \\ $\bullet$ \\ $\bullet$ \\ \color{black} \\ \apple \\ 10.10.3

\author{Daniel Krah}
% \rfoot{Compiled on \today\ at \currenttime}
% \cfoot{}
% \lfoot{Page \thepage}
% \date{1.6.2015}

\begin{document}
% \AddToShipoutPicture{\BackgroundPic}
\maketitle%
\newpage%
 \tableofcontents%
% \newpage
%==================================================================================
% \begin{center}
% \textbf{Vorwort}
% \end{center}



%
% \input{hardware.tex}







%  To Do
% \input{todo.tex}
\newpage
\section{Überblick}


% \begin{displaymath}
%   E = \frac{m_{0} c^{2}}{\sqrt{1-v^{2}/c^{2}}}
% \end{displaymath}

% \begin{luacode}
%   for x=1,600 do
%     tex.print(x+2)
%   end
% \end{luacode}
\subsection{Benötigt:}
\subsubsection{Hardware}
  \begin{enumerate}
    \item Server mit schnellem garantiertem Upload\\
          In diesem Fall eine Dedibox SC mit 2,5 Gbit \\
          (ca 380 Mbit Upload dauerhaft verfügbar)
    \item Einen Uplink ans Backbone des Freifunk Rheinland da der Prozessor zu schwach ist um mehr als 35 Mbit über openVPN zu drücken.  
  \end{enumerate}

\subsubsection{Software}
Softwareseitig werden folgende Pakete/Kernelmodule verwendet:
\begin{enumerate} 
   \item  Ubuntu 16.04 LTS 
   \item  Batman-adv (Kernelmodul -> Einfach laden)
   \item  isc-dhcpd (DCHP-Server für IPv4 Adressen)
   \item  radav (DHCP-Server für IPv6)
\end{enumerate}

% \subsection{Vorraussetzung}

% \begin{itemize} 
%    \item Netzwerkinterface ins Internet eth0
%    \item Netzwerkinterface für VPN mesh-vpn
%    \item Netzwerkinterface für Batman: bat0
%    \item Client IPv4 Netz 10.185.8.0/21 
% \end{itemize}

\section{Tunneldigger}
\subsection{Was ist das Ziel ?}

\monocodebox{sh}{ifconfig}{./command/ifconfigVpnBatTD.sh}{false}{1}{9999999}
\monocodebox{sh}{brctl}{./command/brctlEmpty}{false}{1}{9999999}



\subsection{Die Tunneldigger Bridge}
\monocodebox{sh}{/etc/network/interfaces.d/tunneldigger}{./etc/network/interfaces.d/tunneldigger}{false}{1}{9999999}

\subsubsection{Starten des Brokers}
\monocodebox{sh}{/srv/tunneldigger/start-broker.sh}{./srv/tunneldigger/start-broker.sh}{false}{1}{9999999}
 
\subsubsection{Beim Aufbau einer Verbindung}
\monocodebox{sh}{/srv/tunneldigger/scripts/session-up.sh}{./srv/tunneldigger/scripts/session-up.sh}{false}{1}{9999999}

\subsubsection{Beim Abbau einer Verbindung}
\monocodebox{sh}{/srv/tunneldigger/scripts/session-pre-down.sh}{./srv/tunneldigger/scripts/session-pre-down.sh}{false}{1}{9999999}

\subsubsection{Beim Boot}
\monocodebox{sh}{/usr/local/bin/bat-startup.sh}{./usr/local/bin/bat-startup.sh}{false}{1}{9999999}


\subsection{Konfiguration von AirVPN}

sudo apt-get install openvpn

Bei Air VPN eine Konfig datei erstellen und  in /etc/openvpn/airvpn.conf speichern.
Als standard biegt die config aber alle routen auf das VPN Interface um. Damit ist der Server hinter einem NAT und per SSH nichtmehr zu erreichen. 
Deshalb sollten die Routen automatisch erstellt werden und die Zeile
route-delay 5 <- bei mir war es in der airvpn-config nicht drin...
auskommentiert und

route-noexec

ergänzt werden.

ein 
sudo service openvpn start 
startet den Tunnel
sudo update-rc.d openvpn defaults
startet den Tunnel bei jeden boot.





% \monocodebox{sh}{/etc/network/interfaces.d/tunneldigger}{./etc/network/interfaces.d/tunneldigger}{false}{1}{9999999}


% \monocodebox{sh}{/srv/tunneldigger/scripts/session-up.sh}{./srv/tunneldigger/scripts/session-up.sh}{false}{1}{9999999}

% \monocodebox{sh}{ifconfig}{./command/ifconfigVpnBatTD.sh}{false}{1}{9999999}











% \input{kap1.tex}



% \input{dummy.tex}
% \input{dummy.tex}
% \input{dummy.tex}
% \input{dummy.tex}

% \input{prog1.tex}



%  die beiden unteren beiden includen

% \input{kap1_vorl.tex}
% \input{kap2_vorl.tex}
% \input{kap3_vorl.tex}
% \input{kap4_vorl.tex}
% \input{kap5_vorl.tex}
% \input{kap6_vorl.tex}
% kapitel 7 im Buch

%   Kapitel 2

% \input{2ndKap7.tex}




%==============================================

% % programmieren 2

% \input{kap8.tex}
% \input{kap9.tex}
% \input{kap10.tex}
% \input{kap11.tex}
% \input{kap12.tex}
% \input{last.tex}
% \input{einfuerung.tex}




%=========================================================================================================================
%
%-------------END
\end{document}



% \begin{Verbatim}[frame=lines,
%        framerule=0.2mm,framesep=3mm,
%        rulecolor=\color{monoorange},
%        fillcolor=\color{monogreen},
%        label=Kapitel 1,labelposition=topline]
%   Name    :   Cubieboard 2
%   Size    :   10 cm x 6 cm
%   CPU     :   Allwinner A20 SoC (2 ARM-Cortex A7-Cores with 1 GHz)
%   GPU     :   Mali-400MP2 (OpenGL ES 2.0/1.1)
%   VPU     :   CedarX (max 2160p (Ultra HD))
%   RAM     :   512MB (Test) / 1GB (Produktion) DDR3
%   CONN    :   2x USB Host, 1x USB On-the-go, 1x CIR, 1x SATA
%   VID-OUT :   HDMI @ 1080p
%   AUD-OUT :   S/PDIF, Headphone, HDMI-Audio
%   AUD-IN  :   Mikrophone, Line-In
%   Storage :   4 GB NAND-Flash, 1x MicroSD
%   Network :   10/100-Ethernet
%   DB-Con. :   96 Pin incl I²C, SPI, LVDS
% \end{Verbatim}








% z23dsdsdsdsd
% \color{red}
% \shiftkey
% \capslockkey
% \ejectkey
% \pencilkey
% \returnkey
% \revreturnkey
% \cubie
% \cubiebig
% \archlinux \\

% \subsection{Notes}
% \infobox{Notes}{
% $\ $\\
%  \centering
% vdpau-sunxi is still in development. \\
% $\ $\\
% $\ $\\
% }{$\ $}



%hjhjj
 % \input{aufgabe29.tex}







% sss



% \newpage
% 18.2.

% klausur

% 10:15 - 11:45

% ssdddsd

% 26. Das Projekt Technischer Kundendienst liegt in der Version 3 im Verzeichnis O:\Paul\ Programmierung1\Projekte \Kapitel5, bzw. in S2T. \\

% a. Kopieren Sie das Projekt in Ihr Verzeichnis. Machen Sie sich mit dem Projekt vertraut und spielen Sie die Anwendung durch. \\
% b. Schauen Sie sich die Klasse Beantworter an. Erklären Sie Ihrem Nachbarn den Ablauf der Methode generiereAntwort() der Klasse Beantworter und den genauen Zweck der Sammlung AntwortMap. \\

% c. Stellen Sie sich vor, eine HashMap hätte für die vorliegende Anwendung nicht zur Verfügung gestanden. Hätte man die Aufgabe auch mit einer ArrayList lösen können? \\
% Überlegen Sie sich mit Ihrem Nachbarn, wie eine solche alternative Lösung im Prinzip aussehen würde. \\


% 27. Der Umgang mit Zeichenketten vom Typ String soll an Hand einiger typischer Aufga- ben durchgespielt werden. Nehmen Sie für die folgenden Aufgaben die Dokumentation der Klasse String zur Hand. \\

% a. Kopieren Sie das Projekt Zeichenketten aus dem O:\Paul\Programmierung1\Projekte\ Kapitel05.
% b. Die Klasse StringBearbeitung enthält zahlreiche Methodenentwürfe, deren Implementie- rung noch fehlt. Die Aufgabe der jeweiligen Methode ist in der Kommentierung be- schrieben. Ergänzen Sie den zugehörigen Programmcode. Verwenden Sie dazu Metho- den der Klasse String. Die notwendigen Informationen entnehmen Sie der Dokumentati- on der Klasse String.
% Testen Sie Ihren Programmcode ausgiebig. \\




% \newpage
% \input{ha-aufgabe1.tex}
% \input{ha-aufgabe2.tex}
% \input{ha-aufgabex.tex}


% \lstset{language=Ruby, basicstyle=\scriptsize}
% \begin{lstlisting}
% **** ÄÜÖ
% new car ; dsgjhdsjkghsjdg sjkhgjsfdhg sdjkfghjfkdshgjfsd hjkfdsghjfsdkhgsd hjkfdsgjksfdhgjfsdh jkdfsghfjdsgh dfsj
% sdgnjsdgds dsgdsjgsk sdghsjghsjg sgjhfjkhsdjkhgjksdfhgjksfdlg hjkdfshgjfdhsg dsghfdskghfdjskhgsdfjkhfdgfdjhgjklfdhgjkfdhgjfdhgjkfdghdkfjghdfjk
% \end{lstlisting}

% \begin{listingsbox}{myjava}{Test üÜöÖäÄ}
% test
% ÜüÖöÄä
% \end{listingsbox}
% Über
